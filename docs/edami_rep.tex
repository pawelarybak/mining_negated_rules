% !TeX encoding = UTF-8
% !TeX spellcheck = en_EN
\documentclass{article}
\newcommand\tab[1][1cm]{\hspace*{#1}}
\usepackage[english]{babel}
\usepackage[utf8]{inputenc}
\usepackage{graphicx}
\usepackage{float}
\usepackage{amsmath}
\usepackage{geometry}
\usepackage{multirow}
\usepackage{listings}
\usepackage{color}
\usepackage{hyperref}
\usepackage{tabularx}
\usepackage{makecell}
\usepackage{pdfpages}
\usepackage{longtable}
\usepackage{diagbox}
\definecolor{codegreen}{rgb}{0,0.6,0}
\definecolor{codegray}{rgb}{0.5,0.5,0.5}
\definecolor{codepurple}{rgb}{0.58,0,0.82}
\definecolor{backcolour}{rgb}{0.95,0.95,0.92}

\lstdefinestyle{mystyle}{
	backgroundcolor=\color{backcolour},   
	commentstyle=\color{black},
	keywordstyle=\color{blue},
	numberstyle=\tiny\color{codegray},
	stringstyle=\color{codepurple},
	basicstyle=\footnotesize,
	breakatwhitespace=false,         
	breaklines=true,                 
	captionpos=b,                    
	keepspaces=true,                 
	%numbers=left,                    
	%numbersep=5pt,                  
	showspaces=false,                
	showstringspaces=false,
	showtabs=false,                  
	tabsize=2
}

\lstset{style=mystyle}
\date{}

\author{Paweł Rybal and Karolina Borkowska}

\title{Negative rule mining\\
	{\large EDAMI project}}

 
 \begin{document}
 	\maketitle
 	\section{Aim of the project}
 	The aim of this project was to implement and test a~negative rule mining algorithm that was designed and described in the article titled ,,Mining Positive and Negative Association Rules: An Approach for confirmed Rules'' by Maria-Luiza Antonie and Osmar R. Za\"{\i}ane.
 	
 	\section{Project assumptions}
 	It was established that the project will be done using Python programming language and tested with various data sets and if possible, those that were used by algorithms original creators.
 	
 	\section{Implemented algorithm}
	At the beginning data is read form a~file and then prepared so that each line in file is considered as one transaction. Also the first line is a~header that describes what each element in line represents.
	
	When dataset is ready proper algorithm starts. Its consecutive steps are following:
	\begin{enumerate}
		\item set $k$ as 1, this is the iteration number;
		\item find all frequent item sets of size 1 and save them in the set $f_{1}$;
		\item initialize set $f_{k-1}$ with items form $f_{1}$;
		\item set $f_{k}$ as empty set;
		\item generate frequent candidates list $ck$ form the $f_{k-1}$ and $f_{1}$ items:
		\begin{itemize}
			\item for each item set in $f_{k-1}$, for each item in $f_{1}$, check whether item is part of item set; if it is not, union of item set and this item become a~new candidate;
		\end{itemize}
		\item for each $ck$'s item set:
		\begin{itemize}
			\item check whether its relative support is above a~given threshold (i.e. is a~frequent item set), if it is so add it to $f_{k}$;
			\item for each split combination of this item set (combinations of subsets of each possible length against rest of the items, subset becomes an antecedent, rest is consequent), check the correlation of the antecedent and consequent:
			\begin{itemize}
				\item it if is above a~given threshold check arisen rules confidence, again if it's above threshold assign it to positive rules set;
				\item if it is above minimum correlation, but the support is too low, negate the antecedent and the consequent, again the new rule is checked for support and confidence, to establish whether it should be added to the set of negative rules;
				\item if it is not, but the correlation is below negative minimal correlation two new rules are made: one with negated antecedent and one with negated consequent; both rules have their confidences check and if they are high enough, they are added to negative rules set.
			\end{itemize}			
		\end{itemize}
		\item change $f_{k-1}$ to have $f_{k}$'s values;
		\item if $f_{k}$ is not empty, or algorithm reached maximum length of rules, go back to point 4.
	\end{enumerate}
	
	\section{Using implemented scripts}
	To run the implemented algorithm \texttt{run.py} script has to be called from \textit{sh} terminal with specified parameters. Those parameters are as follows:
	\begin{table}[H]
		\centering
		\begin{tabular}{r|c c c c}
		name&description&parameter&type&obligatory\\
		\hline
		file\_path&Path to data file&--file, -f&string&yes\\
		\hline
		min\_supp&Minimal support&--minsupp, -s&float&yes\\
		\hline
		min\_conf&Minimal confidence&--minconf, -c&float&yes\\
		\hline
		min\_corr&Minimal correlation coefficient&--mincorr, -d&float&no (default = 0.5)\\
		\hline
		max\_len&Maximal rule length&--length, -l&integer&no (default = none)\\
		\hline
		verbose&\makecell{Print to output some info\\ during algorithm work}&--verbose, -v& none &no\\
		\end{tabular}
	\caption{Parameters of \texttt{run.py} script}
	\end{table}
	
	
	\section{Tests}
	\subsection{Datasets}
	Unfortunately due to uncertainty of what type of preprocessing was done on the dataset used in the original research by Maria-Luiza Antonie and Osmar R. Za\"{\i}ane, it was impossible to compare results of this implementation with those given in the article. Six other data sets were used to test the algorithm: 
	\newpage
		
		\begin{longtable}{ p{.15\textwidth} | p{.30\textwidth}| p{.15\textwidth}| p{.30\textwidth} }
			\centering
			\label{res_enh}
			dataset & description & size & classes\\
			\hline
			\hline
			adults&data extracted from the census bureau database & 32561 entries& \makecell{age\\workclass\\ education\\ marital-status\\ occupation\\ relationship\\ race\\ sex\\ capital-gain\\ capital-loss\\ hours-per-week\\ native-country}\\
			\hline
			contraception&subset of the 1987 National Indonesia Contraceptive Prevalence Survey & 1473 entries&\makecell{Wifes-age\\Wifes-education\\Husbands-education\\Number-of-children-ever-born\\Wifes-religion\\Wifes-now-working?\\Husbands-occupation\\Standard-of-living-index\\Media-exposure\\Contraceptive-method-used}\\
			\hline
			mushrooms& mushroom records drawn from The Audubon Society Field Guide to North American Mushrooms (1981) & 8124 entries&\makecell{decision\\cap-shape\\cap-surface\\cap-color\\bruises\\odor\\gill-attachment\\gill-spacing\\gill-size\\gill-color\\stalk-shape\\stalk-root\\stalk-surface-above-ring\\stalk-surface-below-ring\\stalk-color-above-ring\\stalk-color-below-ring\\veil-type\\veil-color\\ring-number\\ring-type\\spore-print-color\\population\\habitat}\\
			\hline
			nursery& derived from a hierarchical decision model originally developed to rank applications for nursery schools & 12960 entries&\makecell{parents\\has\_nurs\\form\\children\\housing\\finance\\social\\health}\\
			\hline
			supermarket& sets of transactions form a~supermarket & 4627 entries&124 classes\\
			\hline
			titanic& Titanic's passengers' data  & 887 entries&\makecell{survived\\pclass\\name\\sex\\age\\siblings/spouses aboard\\parents/children aboard\\fare}\\
			\hline
			\caption{Datasets description}
		\end{longtable} 
	\subsection{Time tests}
	To check whether it is possible to conduct research using presented algorithm and to check its time consumption, for each dataset script was run ten times with support, confidence and correlation coefficient equal to 0.5. This numbers were chosen due to fact that some datasets take too long with smaller numbers, but this configuration at least provides some rules and iterates over all of datasets. For this experiment it is enough, in the next section all datasets will be tested with specialized settings. Results are shown below (datasets are shown form the smallest to the largest):
	\begin{table}[H]
		\centering
		\begin{tabular}{r||c |c |c }
			Dataset&average time&rules found&longest rule\\
			\hline
			\hline
			titanic&0.2311s&1081&4\\
			\hline
			contraception&0.0549s&0&0\\
			\hline
			supermarket&0.3056s&0&0\\
			\hline
			mushrooms&19.1735s&504&4\\
			\hline
			nursery&4.3349s&104&3\\
			\hline
			adult&3.0311s&0&0\\
		\end{tabular}
		\caption{Average times of conducting an experiment on a given dataset}
	\end{table}
	As it was suspected, larger data sets take longer time to compute. This was concluded considering that some datasets do not produce rules, hence time taken to iterate over \textit{adult} dataset takes longer then \textit{contraception} or \textit{supermarket} proves this, as much as time taken for \textit{mushrooms} and \textit{titanic}, that produce some rules. Further sections will consider individual datasets' tests.
	
	
	\subsubsection{Adult dataset}
	\textit{Adult} dataset was tested with settings form table \ref{as} and the results are shown in table \ref{ar}.
	
	\begin{table}[H]
		\centering
		\label{as}
		\begin{tabular}{c |c |c|c|c|c}
			support&confidence&\makecell{correlation\\coefficient}&average time&rules found&longest rule\\
			\hline
			\hline
			0.3&0.3&0.5&87.0020s&532&6\\
			\hline
			0.2&0.2&0.5&203.3814s&1304&7\\
		\end{tabular}
		\caption{Settings used for testing on \textit{adult} dataset}
	\end{table}
	\begin{table}[H]
		\centering
		\label{ar}
		\begin{tabular}{r|c |c}
			rule type& is present & rule example\\
			\hline
			\hline
			x$\rightarrow$ y & yes & \makecell{'race= White', 'native-country= United-States',\\ 'relationship= Husband', 'capital-loss=none' $\rightarrow$\\ 'marital-status= Married-civ-spouse', 'sex= Male'\\ (support: 10315, confidence: 0.9990305380513815)}\\
			\hline
			$\neg x\rightarrow y$ & yes &\makecell{'$\neg$capital-gain=none', '$\neg$sex= Female' $\rightarrow$\\ 'marital-status= Married-civ-spouse',\\ 'native-country= United-States', 'relationship= Husband'\\ (support: 2090, confidence: 0.6947368421052632)}\\
			\hline
			$x\rightarrow \neg y $& yes &\makecell{'sex= Female', 'native-country= United-States',\\ 'capital-gain=none' $\rightarrow$\\ '$\neg$relationship= Husband'\\ (support: 9113, confidence: 0.9998902666520355)}\\
			\hline
			$\neg x\rightarrow \neg y $& no &---------------\\ 
		\end{tabular}
		\caption{Results for \textit{adult} dataset for setting in a previous table}
	\end{table}
	
	This result prove that both ,,normal'' rules  (x$\rightarrow$ y) are produced by algorithm as wall as those with one element negated. The negated rules are a bit tautological in their nature but are true nevertheless.
	
	Lowering minimal support and confidence to 0.2 increased number of rues to 1304 an the longest one was of size 7.
 	
 	\subsubsection{Contraception dataset}
 	\textit{Contraception} dataset was tested with settings form table \ref{cs} and the results are shown in table \ref{ar}.
 	
 	\begin{table}[H]
 		\centering
 		\label{cs}
 		\begin{tabular}{c |c |c|c|c|c}
 			support&confidence&\makecell{correlation\\coefficient}&average time&rules found&longest rule\\
 			\hline
 			\hline
 			0.3&0.3&0.5&0.2183s&6&3\\
 			\hline
 			0.1&0.1&0.5&13.9701s&18&5\\
 		\end{tabular}
 		\caption{Settings used for testing on \textit{contraception} dataset}
 	\end{table}
 	\begin{table}[H]
 		\centering
 		\label{cr}
 		\begin{tabular}{r|c |c}
 			rule type& is present & rule example\\
 			\hline
 			\hline
 			x$\rightarrow$ y & yes & \makecell{'Wifes-education=very-high' $\rightarrow$\\'Media-exposure=Good', 'Husbands-education=very-high'\\ (support: 577, confidence: 0.9306759098786829)}\\
 			\hline
 			$\neg x\rightarrow y$ & no &---------------\\
 			\hline
 			$x\rightarrow \neg y $& no &---------------\\
 			\hline
 			$\neg x\rightarrow \neg y $& yes & \makecell{'$\neg$Husbands-education=very-high' $\rightarrow$\\ '$\neg$Standard-of-living-index=4', '$\neg$Wifes-education=very-high',\\ '$\neg$Number-of-children-ever-born=three-to-five',\\ '$\neg$Wifes-age=early20s'\\ (support: 575, confidence: 0.34608695652173915)}\\ 
 		\end{tabular}
 		\caption{Results for \textit{contraception} dataset for setting in a previous table}
 	\end{table}
 	
 	In the first test (0.3 support and confidence) produces only 6 rules and all of them of first type (everything positive). Lowering minimal values produces additional negative rules. This proves that it is possible to achieve fourth type of rules ($\neg x\rightarrow \neg y$).
	 
	\subsubsection{Mushrooms dataset}
	\textit{Mushrooms} dataset was tested with settings form table \ref{ms} and the results are shown in table \ref{mr}. Due to time restriction and the fact that all tests were run on simple PC, this set was tested only with high minimum value settings. Testing this dataset was stopped when lower minimal values (0.3 support and confidence) after 17 minutes into first iteration.
	
	\begin{table}[H]
		\centering
		\label{ms}
		\begin{tabular}{c |c |c|c|c|c}
			support&confidence&\makecell{correlation\\coefficient}&average time&rules found&longest rule\\
			\hline
			\hline
			0.6&0.6&0.5&2.3888s&56&4\\
			\hline
			0.5&0.5&0.5&18.3183s&504&6\\
		\end{tabular}
		\caption{Settings used for testing on \textit{mushrooms} dataset}
	\end{table}
	\begin{table}[H]
		\centering
		\label{mr}
		\begin{tabular}{r|c |c}
			rule type& is present & rule example\\
			\hline
			\hline
			x$\rightarrow$ y & yes &\makecell{'ring-number=one', 'stalk-surface-below-ring=smooth',\\ 'gill-attachment=free' $\rightarrow$\\ 'veil-type=partial'\\ (support: 4304, confidence: 1.0)} \\
			\hline
			$\neg x\rightarrow y$ & yes &\makecell{'~decision=edible', '~gill-attachment=free' $\rightarrow$\\ 'bruises=no', 'veil-type=partial'\\ (support: 19, confidence: 0.9473684210526315)}\\
			\hline
			$x\rightarrow \neg y $& yes &\makecell{'stalk-surface-below-ring=smooth', 'veil-color=white',\\ 'veil-type=partial', 'gill-attachment=free' $\rightarrow$\\ '~bruises=no'\\ (support: 4744, confidence: 0.6405986509274874)}\\
			\hline
			$\neg x\rightarrow \neg y $& no &--------------- \\ 
		\end{tabular}
		\caption{Results for \textit{mushrooms} dataset for setting in a previous table}
	\end{table}
	
	Mushroom data set does not produce double negative rules ($\neg x\rightarrow \neg y $), this can be due to high support and confidence values, but again it takes too long time to test that theory.
	
	\subsection{Nursery dataset}
	\textit{Nursery} dataset was tested with settings form table \ref{ns} and the results are shown in table \ref{nr}
	
	\begin{table}[H]
		\centering
		\label{ns}
		\begin{tabular}{c |c |c|c|c|c}
			support&confidence&\makecell{correlation\\coefficient}&average time&rules found&longest rule\\
			\hline
			\hline
			0.3&0.3&0.5&0.1454s&0&0\\
			\hline
			0.1&0.1&0.5&4.1125s&104&3\\
			\hline
			0.05&0.05&0.5&26.0353s&702&4\\
		\end{tabular}
		\caption{Settings used for testing on \textit{nursery} dataset}
	\end{table}
	\begin{table}[H]
		\centering
		\label{nr}
		\begin{tabular}{r|c |c}
			rule type& is present & rule example\\
			\hline
			\hline
			x$\rightarrow$ y & yes &\makecell{'finance=convenient', 'health=not\_recom' $\rightarrow$\\ '=not\_recom'\\ (support: 2160, confidence: 1.0)} \\
			\hline
			$\neg x\rightarrow y$ & no &---------------\\
			\hline
			$x\rightarrow \neg y $& no &---------------\\
			\hline
			$\neg x\rightarrow \neg y $& yes &\makecell{~health=not\_recom', '~=priority' $\rightarrow$\\ '~parents=usual'\\ (support: 4374, confidence: 0.7816643804298126)} \\ 
		\end{tabular}
		\caption{Results for \textit{nursery} dataset for setting in a previous table}
	\end{table}
	Setting 0.1 as minimal support and confidence yields double positive (x$\rightarrow$ y) and double negative ($\neg x\rightarrow \neg y $) rules. Lowering requirement does not change that.
	
	\subsection{Supermarket dataset}
	\textit{Supermarket} dataset was tested with settings form table \ref{ss} and the results are shown in table \ref{sr}
	
	\begin{table}[H]
		\centering
		\label{ss}
		\begin{tabular}{c |c |c|c|c|c}
			support&confidence&\makecell{correlation\\coefficient}&average time&rules found&longest rule\\
			\hline
			\hline
			0.3&0.3&0.5&6.0283s&46&3\\
			\hline
			0.2&0.2&0.5&112.5376s&229&4\\
		\end{tabular}
		\caption{Settings used for testing on \textit{supermarket} dataset}
	\end{table}
	\begin{table}[H]
		\centering
		\label{sr}
		\begin{tabular}{r|c |c}
			rule type& is present & rule example\\
			\hline
			\hline
			x$\rightarrow$ y & yes &\makecell{'finance=convenient', 'health=not\_recom' $\rightarrow$\\ '=not\_recom'\\ (support: 2160, confidence: 1.0)} \\
			\hline
			$\neg x\rightarrow y$ & no &---------------\\
			\hline
			$x\rightarrow \neg y $& no &---------------\\
			\hline
			$\neg x\rightarrow \neg y $& yes &\makecell{~health=not\_recom', '~=priority' $\rightarrow$\\ '~parents=usual'\\ (support: 4374, confidence: 0.7816643804298126)} \\ 
		\end{tabular}
		\caption{Results for \textit{supermarket} dataset for setting in a previous table}
	\end{table}
	Only two types of rules are yielded. Lowering requirement does not change that.
	
	\subsection{Titanic dataset}
	\textit{Titanic} dataset was tested with settings form table \ref{ts} and the results are shown in table \ref{tr}
	
	\begin{table}[H]
		\centering
		\label{ts}
		\begin{tabular}{c |c |c|c|c|c}
			support&confidence&\makecell{correlation\\coefficient}&average time&rules found&longest rule\\
			\hline
			\hline
			0.3&0.3&0.5&0.0797s&21&4\\
			\hline
			0.2&0.2&0.5&0.1512s&22&4\\
		\end{tabular}
		\caption{Settings used for testing on \textit{titanic} dataset}
	\end{table}
	\begin{table}[H]
		\centering
		\label{tr}
		\begin{tabular}{r|c |c}
			rule type& is present & rule example\\
			\hline
			\hline
			x$\rightarrow$ y & yes &\makecell{'Sex=male' $\rightarrow$\\ 'Parents/Children Aboard=0', 'Survived=0'\\ (support: 573, confidence: 0.6980802792321117)} \\
			\hline
			$\neg x\rightarrow y$ & no &\makecell{'~Survived=1' $\rightarrow$\\ 'Sex=male'\\ (support: 545, confidence: 0.8495412844036697)}\\
			\hline
			$x\rightarrow \neg y $& no &\makecell{'Parents/Children Aboard=0', 'Survived=0' $\rightarrow$\\ '~Sex=female'\\ (support: 441, confidence: 0.9047619047619048)}\\
			\hline
			$\neg x\rightarrow \neg y $& yes &\makecell{'~Sex=female' $\rightarrow$\\ '~Survived=1'\\ (support: 573, confidence: 0.8097731239092496)} \\ 
		\end{tabular}
		\caption{Results for \textit{titanic} dataset for setting in a previous table}
	\end{table}
	This experiment proves that in one dataset all types of rules can be present, given the right set.
	
	\section{Conclusions}
	Implementation of this algorithm and tests prove that proposed method works. It has its downsides (producing rules as: male $\rightarrow$ not wife), but they are hardly only this algorithms problems. Chosen datasets prove all types of rules can be yielded, give right conditions.
\end{document}





